\documentclass[a4paper,12pt]{report}
\usepackage{polski}
\usepackage[polish]{babel}
\usepackage[utf8]{inputenc}
\usepackage{lastpage}
\usepackage{fancyhdr}
\usepackage{titlesec}
\usepackage{listings}
\usepackage{color}
\usepackage{framed}
\usepackage{etoolbox}
\usepackage{graphicx}
\usepackage{indentfirst}
\usepackage{hyperref}
\hypersetup{
    colorlinks=true,
    linkcolor=black,
    filecolor=black,
    urlcolor=blue,
}

\urlstyle{same}

\makeatletter
\patchcmd{\@verbatim}
  {\verbatim@font}
  {\verbatim@font\small}
  {}{}
\makeatother

\titleformat{\chapter}
  {\normalfont\LARGE\bfseries}{\thechapter}{1em}{}
\titlespacing*{\chapter}{0pt}{0.5ex}{0.5ex}

\title{8mu \\ \large Specyfikacja funkcjonalna}
\author{Krzysztof Piekarczyk \\ \small 288277}
\date{25 marca 2020}

\pagestyle{fancy}
\fancyhf{}
\rhead{Krzysztof Piekarczyk 288277}
\lhead{8mu: Specyfikacja funkcjonalna}
\rfoot{Strona \thepage \space z \pageref{LastPage}}

\begin{document}
\maketitle

\tableofcontents
\thispagestyle{fancy}

\chapter{Wstęp}
\thispagestyle{fancy}

\section{Cel dokumentu}
Celem tego dokumentu jest przedstawienie funkcjonalności i obsługi programu ''8mu,,.

\section{Cel projektu}
Celem projektu jest stworzenie interpretera języka CHIP-8, pozwalającego na uruchamianie napisanych w nim programów.

\section{Użytkownik końcowy}
Użytkownikiem końcowym są ludzie chcący zagrać w najstarsze gry, z których wiele zostało przeportowane na CHIP-8, ponadto każdy, kto chce spróbować swoich sił w pisaniu gier na CHIP-8.

\section{Uzasadnienie nazwy}
Jako, że interpreter CHIP-8 opiera swoją funkcjonalność na prostej maszynie wirtualnej, bardzo często jest używany jako pierwszy krok do świata emulacji. Postanowiłem więc nadać programowi nazwę ''8mu,, która jest zlepkiem słów CHIP-\textbf{8} i e\textbf{mu}lacja, czytane jako ''ejt-mu,,.

\chapter{Uruchomienie programu}
\thispagestyle{fancy}
Program uruchamiany będzie z linii komend: \verb+./8mu <ścieżka>+. Na linii komend zostaną wypisane aktualne kroki w inicjalizacji programu, a po zakończonej inicjalizacji wywołane zostanie osobne okno, które służyć będzie za ekran interpretera CHIP-8.

Program może być uruchomiony również bez podania żadnego argumentu. Zostanie wtedy uruchomiony moduł diagnostyczny, który przeprowadzi testy funkcjonalności i poinformuje o ich wynikach.

\chapter{Dane}
\thispagestyle{fancy}

\section{Dane wejściowe}
Danymi wejściowymi są pliki binarne zawierające kod zgodny ze specyfikacją języka CHIP-8. To rozwiązanie będzie implementowane w oparciu o te źródła:
\begin{itemize}
    \item referencja techniczna: \href{https://github.com/mattmikolay/chip-8/wiki/CHIP\%E2\%80\%908-Technical-Reference}{https://github.com/mattmikolay/chip-8/wiki/CHIP-8-Technical-Reference}
    \item zestaw instrukcji: \href{https://github.com/mattmikolay/chip-8/wiki/CHIP\%E2\%80\%908-Instruction-Set}{https://github.com/mattmikolay/chip-8/wiki/CHIP-8-Instruction-Set}
\end{itemize}

Ponadto, użytkownik może wprowadzać bodźce dla interpretowanego programu poprzez klawisze ''1, 2, 3, 4, Q, W, E, R, A, S, D, F, Z, X, C, V,, a więc siatkę 4x4 na klawiaturze.

\section{Dane wyjściowe}
Program wyświetla informację o inicjalizacji oraz ekran interpretera, który jest kontrolowany przez interpretowany program. Poza tym, program nie tworzy żadnych nowych plików.

\chapter{Sytuacje wyjątkowe}
\thispagestyle{fancy}
Możliwe jest wystąpienie sytuacji, w której w pliku wejściowym, znajdzie się kod operacyjny, który jest niepoprawny lub niezaimplementowany. W wypadku wyłapania takiego kodu na fazie inicjalizacji program zostanie przerwany i zostanie wyświetlona informacja o błędnym kodzie, wraz z pozycją tego kodu w pliku.

\chapter{Testowanie}
\thispagestyle{fancy}
Zostanie zaimplementowany komponent diagnostyczny systemu który będzie sprawdzał, czy kody operacyjne są poprawnie wykonywane, będzie on jednak wymagał wizualnego potwierdzenia poprawnej funkcjonalności pod względem kodów dotyczących wyświetlania.

\end{document}